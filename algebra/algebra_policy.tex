\documentclass[a4paper,12pt]{article}

\usepackage[top = 2.5cm, bottom = 2.5cm, left = 2.5cm, right = 2.5cm]{geometry} 

\usepackage[utf8]{inputenc}
\usepackage[czech]{babel}

\usepackage{multirow}
\usepackage{booktabs}

\usepackage{graphicx}
\usepackage{amsfonts}
\usepackage{amsmath}

\newtheorem{definition}{Definice}[section]
\newtheorem{corollary}{Poznámka}[definition]

\usepackage{syntax}

\usepackage{verbatim}
\usepackage[usenames,dvipsnames]{xcolor}
\usepackage{listings}
\usepackage{color}

\lstset{language=C++,
        basicstyle=\ttfamily,
        keywordstyle=\color{blue}\ttfamily,
        morekeywords={constexpr},
        stringstyle=\color{red}\ttfamily,
        commentstyle=\color{gray}\ttfamily
}

\usepackage{setspace}
\setlength{\parindent}{0in}
\setlength{\parskip}{.5em}

\begin{document}

\section{Kontejnery s jednoduchými souřadnicemi}

\begin{definition}[Paměť]
Paměť je $\mathbb{N}_0$.
\end{definition}

\begin{definition}[Cesta a univerzum cest]
Cesta $p$ je slovo čísel z $\mathbb{N}_0$, univerzum cest budeme značit $P$.
\end{definition}

\begin{definition}[Adresový prostor]
Adresový prostor $A$ je taková podmnožina univerza cest $P$, že $\forall X n Y \in A \wedge n > 0 \wedge X,Y \in P: \exists Y' \in P: X (n - 1) Y' \in A$. (Neboli: každá větev cest je indexovaná od nuly a nejsou přeskočeny žádné indexy)
\end{definition}

\begin{definition}[Velikost prvku]
Velikost prvku daného adresového prostoru $A$ je funkce $\mu_A: \mathbb{N}_0 \to \mathbb{N}_0$.
\end{definition}

\begin{definition}[Offset prvku]
Offset prvku daného adresového prostoru $A$ je funkce $\delta_A: A \to \mathbb{N}_0$, $\delta(a) = \sum\limits_{n=0}^{a-1} \mu_A(n)$.
\end{definition}

\begin{definition}[Adresace]
Mějme dán adresový prostor $A$, potom pro adresový prostor $M \subseteq A$ se zobrazení $\alpha: M \to \mathbb{N}_0$ nazývá adresace (v adresovém prostoru $A$).
\end{definition}

\begin{corollary}[O adresaci]
Adresace může mít následující vlastnosti:
\begin{itemize}
    \item konečnost
    \item monotonie
    \item spojitost
    \item prostost
    \item rostoucnost/klesajícnost
\end{itemize}
\end{corollary}

\begin{definition}[Kontejner]
    Kontejnerem budeme nazývat spojitou (prostou) rostoucí adresaci.
\end{definition}

\begin{definition}[Array]
    Array je takový kontejner v daném adresovém prostoru $A$, že $\mu_A$ je konstantní na jeho definičním oboru $M$ a $M$ je množina obsahující pouze slova délky $1$.
\end{definition}

\begin{definition}[Tuple]
    Tuple je takový kontejner v daném adresovém prostoru $A$, že jeho definičním oborem $M$ je množina obsahující pouze slova délky $1$.
\end{definition}

\subsection{Runtime kontejnery}

\begin{definition}[Univerzum runtime]
    Univerzem runtime budeme nazývat nějakou nekonečnou množinu jevů $R$.
\end{definition}

\begin{definition}[Adresace v runtime]
    Adresací v runtime $\alpha_R$ budeme nazývat zobrazení $R \to U$, kde $R$ je univerzum runtime a $U$ je množina adresací, kde každá adresace je v nějakém adresovém prostoru $A_r$ závislém na runtime $r$.
\end{definition}

\begin{definition}[Statický kontejner]
    Statický kontejner je taková adresace v runtime $s_R$ univerza runtime $R$, že je konstantní a její hodnotou je kontejner.
\end{definition}

\begin{definition}[Dynamický kontejner]
    Dynamický kontejner je taková adresace v runtime $d_R$ univerza runtime $R$, že jejími hodnotami jsou kontejnery a pro aspoň jednu dvojici runtime je její hodnota různá.
\end{definition}

\begin{definition}[Array v runtime]
    Array v runtime je takový statický kontejner $a_R$, jehož hodnotou je vždy array a $\mu_{A_r}$ je na definičních oborech jednotlivých array konstantní s nějakou hodnotou $a$ konstantní vzhledem k runtime $r$.
\end{definition}

\begin{definition}[Tuple v runtime]
    Tuple v runtime je takový statický kontejner $t_R$, jehož hodnotou je vždy tuple a $\mu_{A_r}$ je na definičních oborech konstantní vzhledem k runtime $r$.
\end{definition}

\begin{definition}[Vektor v runtime]
    Vektor v runtime je takový dynamický kontejner $v_R$, jehož hodnotami jsou array vždy v nějakém adresovém prostoru $A_r$, na jejich definičních oborech je $\mu_A$ vždy konstantní a vždy stejná bez ohledu na $r$ a množina všech definičních oborů zmíněných array je $\{\{0,1,\dots n\}; n \in N_0\}$.
\end{definition}

\subsection{Kontejnerová algebra}

\begin{definition}[Abstraktní kontejner]
    Abstraktní kontejner je zobrazení, které každému runtime přiřadí adresový prostor
\end{definition}

\begin{definition}[Array a vektor tagy]
    Array (a vektor) tag je abstrakcí nad danou strukturou značící všechny array (či vektory) v daném univerzu runtime.
\end{definition}

\begin{definition}[Algebra kontejnerů]
    TODO formální definice

    \begin{itemize}
        \item $\cdot$ má sémantiku rozdělení kontejneru na sub-kontejnery, násobí cesty
        \begin{itemize}
            \item alespoň jeden z operandů musí být abstraktní
        \end{itemize}
        \item $+$ má sémantiku řazení za sebe, řadí cesty za sebe, pravého operandu zvětšeny v dominantním rozměru (první znak) o maximum z levého operandu plus jedna (nezvětší, pokud maximum není dobře definováno)
        \begin{itemize}
            \item oba operandy musí být stejného typu (abstraktní $\times$ konkrétní)
        \end{itemize}
    \end{itemize}
\end{definition}

\begin{corollary}[Úplnost algebry kontejnerů]
    TODO: mělo by to platit

    Algebra kontejnerů pokrývá všechny netriviální kontejnery
\end{corollary}

\subsection{View}

\begin{definition}[View]
    View je zobecněním kontejneru. Nemá omezení na spojitost a rostoucnost, je však stále monotónní v každé jedné souřadnici (ne však napříč souřadnicemi).
\end{definition}

\begin{definition}[Reverse, shift a align]
    Funkce \textbf{reverse} obrací indexaci v nejvyšší souřadnici, \textbf{shift} posouvá všechny prvky (obrazy adresace) o zadanou konstantu, \textbf{align} posune danou strukturu (view či kontejner) na násobek dané konstanty.
\end{definition}

\section{Mapy}

\begin{definition}[Mapa]
    Mapa je zobecněním view, kde je upuštěno od monotonie a prostoty. Do této kategorie patří vše myslitelné i nemyslitelné.
\end{definition}

\end{document}
